
\chapter{Conclusion}

Ce mémoire a présenté un système de module spécialisé pour les systèmes
distribués. Il permet la conception d'applications qui exploitent la diffusion
transparente de
modules entre les nœuds. Les appels RPC et la migration de code mobile entre
des nœuds de nature différente sont possibles. Chaque module diffusé a un nom
unique basé sur l'URL du dépôt de code qui permet de le récupérer. La diffusion
se fait au moyen du nom des identifiants qui contient l'URL du module. La façon de nommer les
modules dans ce système est similaire à celui du langage de programmation Go.

Nous avons commencé par exposer les limitations du système Termite Scheme
existant par des expérimentations dans plusieurs situations. Dans le contexte
purement interprété, la migration fonctionnait correctement même dans le cas ou
le nœud de destination ne connait pas le code de l'agent mobile.  C'est dans le
contexte où les applications de chaque nœud sont compilées que la migration de
tâche est un défi, car elle requiert la présence du code compilé de l'agent
mobile sur l'ensemble des nœuds qui peuvent avoir des architectures et un
système d'exploitation différents.

Nous avons exploré plusieurs méthodes de chargement automatique des modules.
Le langage Scheme (que ce soit R5RS, R6RS ou R7RS) permet le chargement de
module à la demande, mais pas simultanément de multiples versions d'un module ce
qui limite énormément la possibilité de mise à niveau d'un service sans
interruption.  Ceci nous a menés à un constat que le nom des identifiants de
module transmis devait être unique au sein du système distribué.

Nous avons ajouté une forme spéciale à Gambit pour permettre la définition de
modules. Ce projet a mené au système de modules spécialisés pour les systèmes
distribués présenté dans ce mémoire. Le résultat est prometteur, il permet le
déploiement de serveurs sur plusieurs machines d'architecture et de systèmes
d'exploitation différents, tel que démontré par nos expériences.

Le système Gambit a été amélioré par l'ajout des modules primitifs et aussi des
modules compatible avec la syntaxe de R7RS. De nouveaux mécanismes de
chargement de modules ont été ajoutés pour garantir l'ordre et le chargement
unique de chaque module.

L'installation et la compilation des modules sont effectuées automatiquement à
la demande.  Les coûts du chargement de module sont amortis après la première
installation et compilation d'un module. L'utilisation interprétée de Termite
n'utilise pas la compilation et l'installation automatique.  La vitesse
d'exécution après la compilation surpasse la version interprétée.  Le facteur
d'accélération observé après la première exécution qui installe les modules
compilés est approximativement de 75x pour le test de 4K, 459x pour le test de
40K et 33x pour le test de 400K, des programmes de taille grandissante.  Le
facteur d'accélération est principalement lié au fait que le code compilé est
plus rapide que celui interprété.  Puisqu'il y a moins de transmission dans le
cas compilé, car les données sont plus compactes que lorsqu'interprété, le
temps de transmission est généralement plus petit. Dans nos tests, la taille de
la sérialisation d'une procédure compilée est environ 400 fois plus petite que
sa version interprétée.

%[TODO: continue here]
L'approche de diffusion des modules est applicable à d'autres langages.  La
coexistence de plusieurs versions d'un module est indépendante du chargement
dynamique et de la sérialisation des objets. Le chargement dynamique est
nécessaire pour développer des applications extensibles, car cela permet le
chargement de bibliothèque durant l'exécution (sans interruption de service).
Les procédures doivent être manipulables comme une donnée et sérialisable pour
pouvoir être transmis à un autre nœud.  Une sérialisation et déserialisation
qui est indépendante de la machine est nécessaire pour la transmission des
objets (nombre, booléen, liste, vecteur, procédures) entre les nœuds d'un
système distribué hétérogène.  La sérialisation doit être similaire à celle qui
est utilisée dans Gambit, par son uniformité sur toutes les plateformes.
L'installation automatique avec le chargement dynamique offre la possibilité de
charger un module qui n'est pas présent sur le nœud courant.  La sérialisation
des procédures contient l'information qui est utilisée dans l'installation
automatique du module qui contient cette procédure.
%[...]

% Il faut que le langage supporte le chargement dynamique de module et
% que les procédures soient des objets manipulables comme des données.
% Il faut aussi que la sérialisation et désérialisation des procédures
% passe par un encodage uniforme et indépendant de la machine.
% L'information encodée doit contenir l'information du module qui contient
% la procédure et un identifiant unique de la procédure. Une
% sérialisation similaire à celle qui est utilisée dans Gambit.

Notre approche de gestion des modules diffère des autres gestionnaires par le
traitement des versions des modules. Plusieurs versions d'un module peuvent
être installées. La gestion des modules dans Racket (\textit{raco}) ne permet
pas d'installer plusieurs versions des modules. Akku, un autre gestionnaire de
modules compatible avec plusieurs implémentations de Scheme a des
caractéristiques similaires à \textit{raco} (le gestionnaire de
\textit{package} de Racket~\cite{Racket}).

L'implémentation actuelle des modules a quelques aspects mineurs à améliorer.
Les macros définies dans le contexte d'un module ont des problèmes d'hygiènes
lors de l'expansion dans un autre module. Le temps d'installation des modules
pourrait être optimisé. L'opération de mise à jour ne permet pas de mettre à jour
les branches qui sont installées.  Les messages d'avertissement
(\textit{warnings}) lors de la compilation des modules sont manquants. Il n'y a
pas de message indiquant l'utilisation d'une liaison non définie.

%[FIXME: update using other branch then master]

En résumé, notre approche de diffusion des modules compilés offre une
plus grande flexibilité et performance à l'exécution. Cela permet la mise à jour de
code d'un nœud distant sans interrompre son service et sans intervention
manuelle de déploiement du nouveau code. Cette approche
est applicable à d'autres langages. La transmission permet d'avoir
une beaucoup plus grande performance d'exécution.

%[TODO: nice ending paragraph]

% - Expérimentation dans Termite
% - Diffusion de module
% - Code mobile compilé
% - Approche par module.
% - Performance
