
\chapter{Conclusion}.

Ce mémoire a présenté un système de module spécialisé pour les systèmes
distribués. Il permet la conception d'applications qui exploite la diffusion de
modules entre les nœuds. Ce qui permet des appels RPC et la migration de code
mobile sur un nœud générique. C'est pour cela que le nommage des modules du
système Scheme, similaire à GoLang.

Nous avons commencé par exposer les limitations du système Termite Scheme par
des expérimentations dans plusieurs situations. Dans le contexte purement
interpréter la migration fonctionnait déjà, même dans le cas ou le nœud
destination ne connait pas le code de l'agent mobile.  C'est dans le contexte
où les applications sur les nœuds sont compilées que la migration de tâche
requiert la présence du code de l'agent mobile. Ce qui nous a menés à un
constat que le nom des identifiants transmit manquait d'information et devait
être unique au sein du système distribué. La structure des modules présents en
était la cause, puisque basée sur R5RS. Nous avons ajouté une forme spéciale à
Gambit pour permettre la création de modules.

Le projet a mené au système de modules spécialisés pour les systèmes distribués
à commencer par des expérimentations avec Termite. Le résultat présent est
prometteur, il permet le déploiement de serveurs sur plusieurs machines
d'architecture et de systèmes d'exploitation différents.

Le système Gambit-C a été amélioré par l'ajout des modules primitifs et aussi
des modules R7RS. De nouveaux mécanismes de chargement de modules ont été
ajoutés pour garantir l'ordre et le chargement unique de chaque module.


L'installation et la compilation des modules sont effectuées automatiquement
lorsque demandé.  Les performances du chargement de module sont amorties après
la première installation et compilation d'un module. L'utilisation interprétée
de Termite n'utilise pas la compilation et l'installation automatique.
La vitesse d'exécution après l'amortissement surpasse la version interprétée.

La majorité du temps de module permet des temps de diffusion et d'exécution
plus courts.




% - Expérimentation dans Termite
% - Diffusion de module
% - Code mobile compilé
% - Approche par module.
% - Performance
