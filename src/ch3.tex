
\chapter{Implémentation des modules}
\label{ch:modules_implementation}
%[FIXME: module manquant]
Ce chapitre traite de notre implémentation des modules dans Gambit qui se veut
le plus portable entre les différentes implémentations de Scheme.  Nous avons
ajouté une syntaxe pour les modules à Gambit parce qu'elle était manquante.
Notre implémentation des modules se base sur la syntaxe de R7RS et utilise des
formes déjà existantes et aussi de nouvelles formes qui ont qui ont été
ajoutées au besoin.  Ces nouvelles formes nous ont permis d'intégrer la syntaxe
des modules R7RS avec les modules primitifs de Gambit. Il y a deux façons
d'écrire des modules.  Les modules primitifs sont utilisés pour implémenter des
fonctionnalités système dans Gambit, par exemple, un module qui implémente une
nouvelle syntaxe.  La syntaxe des modules R7RS est utilisée lorsqu'un module se
veut compatible avec d'autres implémentations du langage de programmation
Scheme supportant R7RS.

Les formats des modules R7RS, construits avec \lstcode{define-library}, contiennent
plusieurs composantes.
\begin{itemize}
  \item L'espace de nom du module qui regroupe toutes les fonctionnalités.
  \item Une liste des modules qui sont utilisés par le module courant.
  \item Une liste des symboles exportés par le module.
\end{itemize}
Il est possible de modifier les identifiants d'un module lors de l'importation et de l'exportation.
L'importation multiple d'un module doit correspondre à un seul chargement.
Pour exprimer les relations entre les modules certaines formes spéciales
ont été ajoutées dans Gambit.  Si les concepteurs de bibliothèques respectent
la syntaxe R7RS, alors il est possible de l'importer dans Gambit. C'est indépendant
du système Scheme pour lequel les bibliothèques ont été écrites.



% Le format des modules nécessite plusieurs composantes.
% Chaque module a son propre espace de nom
% disjoint des autres modules. Un liste d'importation
% qui indique les dépendances entre les modules et permet
% l'interaction entre les modules. Certaines formes spéciales
% ont été ajouté dans pour pouvoir exprimer les relation entre
% les modules. Les modules R7RS permette de spécifier les identifiant
% d'un module et de les renommer. Les relations entre les modules doivent
% pouvoir être exprimées.  La liste des modules requis par le module
% courant et les modules publiés par le module courant.

% - Créer un espace pour les modules.
% - Spécifier les dépendence. (2)
% - Spécifier les paramètre de compilation.
\section{La notation des fichiers dans Gambit}
%[TODO: notation ~~]
Gambit utilise une notation qui est compatible avec l'environnement du
système hôte. Il permet de référer au répertoire maison de l'utilisateur
et aux répertoires d'installation.

Un chemin est une chaîne de caractère qui dénote une ressource.
Le séparateur des composantes d'un chemin dépend du système
d'exploitation, Linux et macOS utilisent `/' alors que MSDOS et
Microsoft Windows utilisent `$\backslash$'.

La notation qui permet de référer au répertoire maison de l'utilisateur courant
est par la notation `\url{~}' qui n'est pas suivit d'un autre `\url{~}'.  Le
nom qui suit `\url{~}' est le nom de l'utilisateur. Par exemple, \url{~USER}
représente le répertoire de l'utilisateur {\tt USER}.

Un répertoire symbolique commence par les caractères `\url{~~}'. S'il n'y a
rien qui suit ces caractères alors ça réfère au répertoire d'installation
central de Gambit.  Sinon ce qui suit le `\url{~~}' est le nom d'un répertoire
symbolique qui sera cherché dans un dictionnaire de répertoires symboliques.
Par exemple, `\url{~~lib}' réfère au répertoire d'installation `\url{lib}' des
bibliothèques système de Gambit. Il est a noté que Gambit permet de redéfinir
l'emplacement des répertoires symboliques par l'utilisation de l'option
d'exécution `\verb|-:~~NAME=DIRECTORY|'.

\section{La forme \lstcode{\#\#namespace}}

Un identifiant est un symbole, utilisé dans le code, qui n'est pas sous un
\lstcode{quote}. Par exemple \lstcode{pair?} est l'identifiant de la procédure
qui teste que le type de donnée est une paire. Il est généralement lié à une
valeur ou une macro. Un symbole est un type de donnée primitif de Scheme.

Les espaces de nom sont gérés avec une forme spéciale propre à Gambit. Cette
forme se nomme \lstcode{##namespace} et permet de lier des identifiants à
d'autres identifiants dans le code. Cela affecte l'ensemble des identifiants
spécifiés dans la forme \lstcode{##namespace}, l'ensemble vide correspond à
tous les identifiants. Cette forme s'applique uniquement aux identifiants de
variable et de macro. Elle ne s'applique pas aux valeurs.  Cette forme
primitive est présente dans Gambit depuis longtemps.  Un espace de nom se
compose de n'importe quelle séquence de caractère terminé par un \lstcode{#}.
Il y a seulement l'espace de nom vide qui est une exception à cette règle et
c'est l'espace de nom par défaut.  Les associations de symboles données par la
forme \lstcode{##namespace} respectent la portée lexicale. Il y a trois types
d'opérations avec les espaces de nom.

Il y a la déclaration d'espaces de nom global qui s'applique à tous les
symboles qui ne contiennent pas de \lstcode{#}. \\[1ex]
\begin{figure}[ht]
  \centering
  \lstset{language={scheme},frame=single}
  \begin{mplisting}{0.6}
(##namespace ("<ns>"))
;; <symbol-name> => <ns><symbol-name>
\end{mplisting}
  \caption{Namespace Global}
  \label{fig:forms->namespace-global}
\end{figure}

Il est possible de spécifier la liste des symboles qui sont affectés par la
déclaration d'espace de nom. À partir de la syntaxe de la figure
\ref{fig:forms->namespace-global}, il suffit d'ajouter les symboles après le nom
de l'espace de nom.\\[1ex]
\begin{figure}[ht]
  \centering
  \lstset{language={scheme},
          frame=single}
  \fontsize{12}{10}
  \begin{mplisting}{0.6}
(##namespace ("<ns>" A B ...))
;; A => <ns>A
;; B => <ns>B
;; ...
\end{mplisting}
  \caption{Namespace Set}
  \label{fig:forms->namespace-set}
\end{figure}

%Chaque symbole dans l'espace de nom peut être renommé à un nouveau symbole en
%spécifier une pair contenant l'association. La forme ressemble à l'exemple
La forme \lstcode{##namespace} permet aussi de lier un identifiant à un
autre identifiant dans un espace de nom donné. Chaque association est marquée
par une paire qui crée un alias entre le premier élément et le second. Par exemple,
la paire \lstcode{(<old> <new>)} remplace \lstcode{<old>} par \lstcode{<ns><new>}.
\\[1ex]
\begin{figure}[ht]
  \centering
  \lstset{language={scheme},
          frame=single}
  \fontsize{12}{10}
  \begin{mplisting}{0.6}
(##namespace ("<ns>" (<old> <new>) ...))
;; <old> => <ns><new>
;; ...
\end{mplisting}
  \caption{Namespace Rename}
  \label{fig:forms->namespace-rename}
\end{figure}

% La syntaxe de la
% forme \lstcode{##namespace} qui inclus toutes les symboles dans l'espace de nom
% marqué par \lstcode{<ns>} est:

% \begin{center}
%   \begin{mplisting}{0.8}
% (##namespace ("<ns>"))
% \end{mplisting}
% \end{center}

% Il est possible de spécifier une liste des symboles qui appartiennent à
% l'espace de nom. Notez que seulement le espace de nom est entre guillemets.

% \begin{center}
%   \begin{mplisting}{0.8}
% (##namespace ("<ns>" sym1 ...))
% \end{mplisting}
% \end{center}


% L'utilisation de la forme \lstcode{##namespace} permet de associer plusieurs
% identifiant dans un espace de nom spécifique.  La procédure \lstcode{hello} est
% défini dans l'espace de nom \lstcode{hello#}.

Cette forme est utilisée pour créer un espace distinct pour chaque module.
Cela permet d'éviter les conflits de nom entre les identifiants des modules.
Chaque module commence par déclarer son espace de nom suivi des définitions
des procédures du module. Les différentes formes d'espace de noms sont données
par les figures \ref{fig:forms->namespace-global}, \ref{fig:forms->namespace-set}
et \ref{fig:forms->namespace-rename}.
\\[1ex]
\begin{figure}[ht]
  \centering
  \lstset{language={scheme},
          frame=single}
  \fontsize{12}{10}
\begin{mplisting}{0.6}
;; hello.scm
(##namespace ("hello#" hi))
(define (hi)
  (display "Hello, world!\n"))
(hi)
\end{mplisting}
  \caption{Module Hello}
  \label{fig:namespace->hello}
\end{figure}

L'exemple \ref{fig:namespace->hello} est un exemple d'utilisation de la forme
\lstcode{##namespace} pour créer un espace pour le module \lstcode{hello}.
La procédure \lstcode{hi} est dans l'espace de nom \lstcode{hello#}.

\section{La forme \lstcode{\#\#demand-module} et \lstcode{\#\#supply-module}}

Le mécanisme de chargement des modules est géré par la forme spéciale
\lstcode{##demand-module}. Cette forme indique au système de charger un module
s'il n'est pas déjà chargé. Cette forme gère le chargement multiple d'un
module. Elle est utilisée pour importer la liste des modules requis par le
module courant.  Le fonctionnement de cette forme est similaire à la procédure
\lstcode{load} avec quelques différences. La forme spécial
\lstcode{##demand-module} génère une expression vide. L'effet de cette
forme agit après la phase d'expansion des macros. Le paramètre passé à
\lstcode{##demand-module} doit être un symbole qui correspond au nom du module.
La procédure \lstcode{load} requiert le chemin complet vers le fichier à
charger.

Il est à noter que l'ordre dans lequel les \lstcode{##demand-module}
apparaissent correspond à l'ordre dont les modules sont visités. Cette forme
est permise partout où une définition de macro est permise.
%[TODO: ##demand-module]
%% Review.

% \begin{center}
%   \begin{mplisting}{0.9}
% (##demand-module A)
% (##demand-module A)
% (##demand-module B)
% \end{mplisting}
% \end{center}

Nous avons ajouté un forme spéciale conjointe au \lstcode{##demand-module}
pour indique le nom symbolique des modules des modules exportés. Cette
forme spéciale est \lstcode{##supply-module}, elle accepte comme paramètre le
nom du module exporté par l'entité courant.  La syntaxe de ces deux formes dans
la figure \ref{fig:syntax->demand/supply-module}.\\
\begin{figure}[ht]
  \centering
  \lstset{language={scheme},
          frame=single}
  \fontsize{12}{10}
  \begin{mplisting}{0.8}
(##demand-module %*\textit{<module-ref>}*)
(##supply-module %*\textit{<module-ref>}*)
\end{mplisting}
  \caption{Syntaxe des formes \lstcode{demand-module} et \lstcode{supply-module}}
  \label{fig:syntax->demand/supply-module}
\end{figure}

\subsection{Les méta informations}
%
Il est utile d'attacher à un
module des informations qui sont accessibles lors de l'expansion
et même la compilation. Ces informations sont spécifiées par la forme
\lstcode{##meta-info}. Cette forme accepte au moins un paramètre qui
correspond au nom de la méta information, le reste des paramètres est la valeur
associée à la méta donnée.

\begin{center}
  \begin{mplisting}{0.5}
(##meta-info %*\textit{<name>}* %*\textit{<value>}*)
(##meta-info %*\textit{<name>}* %*\textit{<value>}* ...)
\end{mplisting}
\end{center}

Les méta informations sont utilisées pour donner des paramètres de compilation
du module. Les différentes méta informations sont \lstcode{cc-options},
\lstcode{ld-options}, \lstcode{ld-options-prelude}, \lstcode{pkg-config}
et \lstcode{pkg-config-path}. Ces méta informations ne sont utiles que pour les modules
compilés.

\begin{itemize}
  \item Les \lstcode{cc-options} sont ajoutés aux options de la commande qui
    invoque le compilateur C.

  \item Les méta informations \lstcode{ld-options} et \lstcode{ld-options-prelude}
    composent les paramètres de la commande qui invoque l'éditeur de lien.
    Les paramètres dans \lstcode{ld-options-prelude} précèdent ceux
    qui sont dans \lstcode{ld-options}.

  \item \lstcode{pkg-config} contient le nom des bibliothèques C à être
    liées au module Scheme. Les options nécessaires pour le compilateur
    C sont déterminées automatiquement par l'utilitaire \lstcode{pkg-config}.

  \item \lstcode{pkg-config-path} ajoute des répertoires à la variable d'environnement
    \lstcode{PKG_CONFIG_PATH} qui est utilisée par l'utilitaire \lstcode{pkg-config}.

\end{itemize}

\section{Implémentation des modules primitifs}

Un module primitif est généralement constitué d'un fichier d'entête avec la
déclaration de l'espace de nom et les définitions de macros et un fichier
contenant les procédures. Dans Gambit les fichiers d'entête sont marqués par un
\lstcode{#} juste avant l'extension, tel que \texttt{angle2/angle2\#.scm}.

\begin{itemize}
  \item \texttt{\textit{<name>}\#.scm} est la structure du nom fichier d'entête.
    Ce fichier contient des déclarations d'espace de nom et des
    définitions de macros.

  \item \texttt{\textit{<name>}.scm} est la structure du nom du fichier qui contient
    les procédures du module.

\end{itemize}

Le nom des fichiers doit correspondre à la dernière partie du nom de module.  Par
exemple, le module primitif \lstcode{angle2} doit inclure les fichiers
\lstcode{angle2/angle2.scm} et généralement \lstcode{angle2/angle2#.scm}.\\
\begin{figure}[ht]
  \lstset{language={scheme},frame=single}
  \fontsize{12}{10}
\begin{mplisting}{0.9}
;; angle2/angle2#.scm
(##namespace ("angle2#" deg->rad rad->deg))
\end{mplisting}\\[3ex]
\begin{mplisting}{0.9}
;; angle2/angle2.scm
(include "angle2#.scm")
(##namespace ("angle2#" factor))
(##supply-module angle2)
(define factor (/ (atan 1) 45))
(define (deg->rad x) (* x factor))
(define (rad->deg x) (/ x factor))
\end{mplisting}
  \caption{Écriture d'un module qui implémente des fonctions de conversions
           entre les angles en degrés et en radian. Ce module est séparé en 2 fichiers.
           Le fichier \texttt{angle2/angle2\#.scm} contient les exportations et
           \texttt{angle2/angle2.scm} contient l'implémentation des
           fonctions.}
  \label{fig:module->angle}
\end{figure}

% XXX: END implementation
%\todo{Continuer ici}

%\begin{center}
%  \begin{figure}[h]
%  \begin{tabular}{|l|l|}
%\hline
%\begin{mplisting}{0.5}
%;; Library
%(library (math)
%  (export fact)
%  (import (rnrs base))
%  (define (fact n)
%    (if (< n 2)
%      1
%      (* n (fact (- n 1))))))
%\end{mplisting} &
%\begin{mplisting}{0.5}
%;; Main program
%(import
%  (rnrs base)
%  (rnrs io simple)
%  (math))
%
%(display (fact 5))
%(newline)
%\end{mplisting}\\\hline
%  \end{tabular}
%\caption{À gauche, il y a un exemple d'une bibliothèque mathématique dans le format R6RS qui implémente
%la fonction factoriel. À droite, un exemple d'importation de la bibliothèques qui utilise la forme
%spéciale \texttt{import}.}
%\end{figure}
%\end{center}

Dans l'exemple de la figure \ref{fig:module->angle}, il y a dans
\lstcode{angle2/angle2.scm} l'inclusion du fichier d'entête
\lstcode{angle2/angle2#.scm} qui ajoute une déclaration redondante de l'espace
de nom dans ce cas.  La déclaration \lstcode{(##namespace ("angle2#"))}
implique l'espace de nom ajouté par l'inclusion du fichier d'entête. Il est
possible que l'espace de nom déclaré dans \lstcode{angle2/angle2#.scm} soit
différent de celui utilisé dans \lstcode{angle2/angle2.scm}.

La forme \lstcode{##namespace} dans l'exemple \ref{fig:module->angle}
s'applique aux identifiants suivants:
\begin{center}
  \lstset{language={scheme},keepspaces=true}
  \begin{mplisting}{0.3}
factor    --> angle2#factor
deg->rad  --> angle2#deg->rad
rad->deg  --> angle2#rad->deg
\end{mplisting}
\end{center}

\subsection{La forme \lstcode{\#\#import}}
%
L'importation des modules est effectuée par la forme \lstcode{##import} qui
effectue deux actions, l'inclusion du fichier \lstcode{<name>#.scm} et un
chargement des définitions.  La forme \lstcode{##import}, comme
\lstcode{##demand-module} s'occupe de trouver l'emplacement du fichier d'entête
à partir du nom du module. Elle génère le \lstcode{##include} du fichier
d'entête s'il existe et un \lstcode{##demand-module} du module.  L'importation
\lstcode{(##import angle2)} est équivalente à:\\
\begin{figure}[ht]
  \centering
  \lstset{language={Scheme},frame=single}
  \fontsize{12}{10}
  \begin{mplisting}{0.9}
(##include "/un/chemin/angle2/angle2#.scm")
(##demand-module angle2)
\end{mplisting}
  \caption{Expansion de \lstcode{(\#\#import angle2)}}
  \label{fig:prim-import->angle2}
\end{figure}


\section{Implémentation des modules R7RS}

Pour que le système de module soit compatible avec d'autres implémentations
de Scheme,  les modules de haut niveau sont définis dans le standard R7RS
Small~\cite{Scheme:R7RS}. Les modules sont définis par la forme
\lstcode{define-library} dont la syntaxe est donnée par la
figure~\ref{fig:syntax->define-library}. Les formes \lstcode{define-library} et
\lstcode{import} sont expansées dans les formes spéciales utilisées par les modules
primitifs. L'élément qui distingue un module primitif et un module R7RS
est donc l'utilisation de la forme \lstcode{define-library}.

\subsection{Expansion du \lstcode{import}}
\label{sec:import-expand}

L'expansion de la forme \lstcode{import} dépend du type de module qui est
importé. L'importation d'un module primitif est différente de l'importation
d'un module R7RS. Gambit permet l'importation d'un module primitif en utilisant
la même forme que pour les modules R7RS. Les capacités du \lstcode{import}
dépendent de sa provenance, s'il est dans un \lstcode{define-library} ou dans
un programme principal. Dans le cas d'un \lstcode{define-library} le
\lstcode{import} supporte l'importation relative, qui est une extension de
Gambit.


\subsubsection{Importation d'un module primitif}
%[FIXME: verify if ##import doesn't support only except rename]
L'importation d'un module primitif limite la syntaxe du \lstcode{import}.  Il
n'est pas possible d'utiliser les extensions \lstcode{only}, \lstcode{except}
et \lstcode{rename} sur un module primitif. Pour utiliser ces extensions, il
faut les métadonnées que les modules R7RS offrent.  La structure modules
primitifs ne contient pas de déclaration qui indique les identifiants
uniformes. Pour avoir l'ensemble des identifiants qu'un module primitif
il faudrait analyser l'ensemble des fichiers du module primitif pour
y extraire les identifiants qui sont exportés. Nous avons choisi
la simplicité.

Le \lstcode{import} R7RS se rabat sur le \lstcode{##import} des modules
primitifs qui ne supporte pas les extensions R7RS.  C'est une extension que
nous avons ajouté dans notre implémentation des modules R7RS pour permet
l'utilisation des modules primitifs dans un contexte R7RS.

Les modules R7RS ont tous la déclaration \lstcode{export} qui donne l'ensemble
des identifiants qui sont exportés. Les modules primitifs sont plus proches de
R5RS avec un mécanisme de chargement sophistiqué. Le \lstcode{import} requiert
la méta information fournit par la déclaration \lstcode{export} dans le
\lstcode{define-library}.

Les modules primitifs font le pont entre R5RS et R7RS.\\
\begin{figure}[ht]
  \lstset{language={scheme},frame=single}
  \fontsize{12}{10}
\begin{mplisting}{0.8}
;; expansion of (import (termite))
(##import termite)
\end{mplisting}
    \caption{Expansion du \texttt{import} d'un module primitif}
    \label{fig:import->expand-primitive}
\end{figure}

\subsubsection{Importation d'un module R7RS}

L'importation d'un module R7RS est expansée en au plus trois parties.
\begin{itemize}

  \item Un \lstcode{##demand-module} qui s'occupe de charger l'implémentation des
    procédures du module.

  \item Une déclaration d'espace de nom qui donne accès aux identifiants que le
    module exporte.

  \item L'implémentation des macros qui sont exportées par le module.
\end{itemize}

L'instruction de chargement du module est générée dans tous les cas qu'un
module définit des procédures. Un module qui ne définit que des macros ne
nécessite pas d'être chargé durant l'exécution seulement dans le contexte
d'expansion des macros. L'importation d'un module R7RS qui ne contient qu'une
déclaration \lstcode{export} ne nécessite pas d'être chargé durant l'exécution.
Ce type de module est utilisé pour exporter les fonctionnalités déjà
implémentées dans Gambit dans un contexte R7RS.

La forme utilisée pour rendre disponible l'ensemble des identifiants importés
est \lstcode{##namespace}. L'ensemble des identifiants importés dépend de la
forme du \lstcode{import}. Par défaut, tous les identifiants exportés par le
module sont importés.  Les opérateurs \lstcode{only} et \lstcode{except}
affectent le nombre d'identifiants importés. Les opérateurs \lstcode{prefix} et
\lstcode{rename} affectent le nom des identifiants.  Dans l'exemple
\ref{fig:import->expand-r7rs}, l'importation inclut l'ensemble des identifiants
exportés par le module. L'ensemble des formes \lstcode{##namespace} générées
par un \lstcode{import} est donné par la
figure~\ref{fig:import->expand-r7rs-namespace}.\\

\begin{figure}[ht]
  \centering
  \lstset{language={scheme},frame=single}
  \fontsize{12}{10}
  \begin{mplisting}{1}
;; expansion of (import (github.com/gambit/hello))
(##demand-module github.com/gambit/hello)
(##namespace ("github.com/gambit/hello#" hi salut))
;; macros
\end{mplisting}
  \caption{L'exemple de l'expansion du \texttt{import} du module R7RS
    \lstcode{github.com/gambit/hello} qui exporte les procédures
    \lstcode{hi} et \lstcode{salut}.}
  \label{fig:import->expand-r7rs}
\end{figure}

\begin{figure}[ht]
  \centering
\vspace*{3ex}
  \lstset{language={scheme},frame=single}
  \fontsize{12}{10}
  \begin{mplisting}{1}
;; (import (only (github.com/gambit/hello) hi))
(##namespace ("github.com/gambit/hello#" hi))

;; (import (except (github.com/gambit/hello) hi))
(##namespace ("github.com/gambit/hello#" salut))

;; (import (prefix (github.com/gambit/hello) m-))
(##demand-module github.com/gambit/hello)
(##namespace ("github.com/gambit/hello#" (m-hi hi) (m-salut salut)))

;; (import (rename (github.com/gambit/hello) (hi howdy)))
(##namespace ("github.com/gambit/hello#" (howdy hi) salut))
\end{mplisting}
  \caption{Différents \lstcode{\#\#namespace} générés par
    l'expansion du \texttt{import} d'un module R7RS.}
  \label{fig:import->expand-r7rs-namespace},
\end{figure}


\subsection{Expansion du \lstcode{define-library}}

La forme \lstcode{define-library} est expansée dans les formes qui composent un
module primitif. Chacune des déclarations de la bibliothèque est utilisée dans
l'expansion du \lstcode{define-library}. La déclaration d'exportation est
valide si tous les identifiants exportés sont distincts. Une déclaration
\lstcode{export} qui exporte un identifiant plusieurs fois cause une erreur de
syntaxe. Les informations sur les identifiants exportés ne sont pas utilisés
lors de l'expansion du
\lstcode{define-library} (figure \ref{fig:define-library->expand}), mais lors de
l'importation de cette bibliothèque.  Les déclarations \lstcode{import} sont
expansées de la même façon que dans des programmes principaux.\\[3ex]

\begin{figure}[ht]
  \centering
  \lstset{language={scheme},frame=single}
  \fontsize{12}{10}
\begin{mplisting}{0.9}
(define-library (hello)
  (import (scheme base) (scheme write))
  (export hi salut)
  (begin
    (define (exclaim msg1 msg2)
      (display msg1)
      (display msg2)
      (display "!\n"))
    (define (hi name) (exclaim "hello " name))
    (define (salut name) (exclaim "bonjour " name))
    ;; it is best for a library to not have side-effects...
    #;(salut "le monde")))
\end{mplisting}\\[4ex]
\begin{mplisting}{0.9}
;; expansion of (define-library (hello) ...)
(##declare (block))
(##supply-module github.com/gambit/hello)
(##namespace ("github.com/gambit/hello#"))
(##namespace ("" define ...))
(##namespace ("" write-shared write display write-simple))
(define (exclaim msg1 msg2)
    (display msg1) (display msg2) (display "!\n"))
(define (hi name) (exclaim "hello " name))
(define (salut name) (exclaim "bonjour " name))
(##namespace (""))
\end{mplisting}
  \caption{Le code source du module
    \lstcode{github.com/gambit/hello} et son expansion.}
  \label{fig:define-library->expand}
\end{figure}

\subsubsection{Extensions de Gambit}

Gambit offre des extensions au \lstcode{define-library} et au \lstcode{import}.
L'importation dans le contexte d'une bibliothèque peut être relative au module
courant. Plusieurs déclarations supplémentaires ont été ajoutées dans la forme
\lstcode{define-library}.

\begin{itemize}
  \item \lstcode{namespace}
  \item \lstcode{cc-options}
  \item \lstcode{ld-options} et \lstcode{ld-options-prelude}
  \item \lstcode{pkg-config} et \lstcode{pkg-config-path}
\end{itemize}

La figure~\ref{fig:relative-import} est un exemple d'importation relative.
L'importation relative part du \lstcode{module-ref} du module courant.  Un
\lstcode{import} de \lstcode{(.. C)} à partir du module \lstcode{(A B)}
correspond à l'importation de \lstcode{(A C)}. Cela permet au sous-module de
tests unitaires de référer au module principal en préservant le
\lstcode{module-ref} avec la version. \\[1ex]

\begin{figure}[ht]
  \centering
  \lstset{language={scheme},frame=single}
  \fontsize{12}{10}
  \begin{mplisting}{0.8}
(define-library (A B)
  (import (.. C))  ;; => (import (A C))
  (import (..))) ;; => (import (A))
\end{mplisting}
  \caption{Exemple d'importation relative du module}
  \label{fig:relative-import}
\end{figure}

La déclaration \lstcode{namespace} permet de forcer l'espace de nom d'un module.
L'utilisation primaire de cette déclaration est l'implémentation de modules qui
exporte les fonctionnalités déjà implémentées dans Gambit. \\[1ex]

\begin{figure}[ht]
  \lstset{language={scheme},
          frame=single}
  \fontsize{12}{10}
  \begin{mplisting}{0.8}
(define-library (scheme case-lambda)
  (namespace "")
  (export
case-lambda
))
\end{mplisting}
  \caption{Implémentation de la bibliothèque système \lstcode{(scheme case-lambda)}.}
  \label{fig:module->scheme/case-lambda}
\end{figure}

Les déclarations \lstcode{cc-options}, \lstcode{ld-options}, \lstcode{ld-options-prelude},
\lstcode{pkg-config} et \lstcode{pkg-config-path} permet d'ajouter des éléments dans les
méta informations respectifs.

\section{Conclusion}

Nous avons ajouté la notion de module dans Gambit pour résoudre le manque de
modularité dans le langage Termite.  Les formes que nous avons ajoutées sont:
\lstcode{##demand-module}, \lstcode{##supply-module}, \lstcode{##import},
\lstcode{import} et \lstcode{define-library}.  Notre implémentation des modules
utilise une forme compatible avec les autres implémentations de Scheme R7RS. En
plus, les extensions que nous avons ajoutées offrent une interface avec les
modules système de Gambit. L'implémentation actuelle des modules est suffisante
pour permettre la diffusion de code, car elle offre des identifiants uniques
pour les modules.

Notre approche supporte la création de modules qui dépendent de bibliothèques
du système d'exploitation.  Cela est réalisé par les extensions de la forme
\texttt{define-library} et par la FFI de Gambit. Le système de module permet
de regrouper de façon logique les fonctionnalités dans un module.
