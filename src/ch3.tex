
\chapter{Implémentation des modules}

Le format des modules nécessite plusieurs composantes.
Chaque module doit avoir sont propre espace de nom
disjoint des autres modules. Un liste d'importation
qui indique les dépendances entre les modules et permet
l'interaction entre les modules. Certaines formes spéciales
ont été ajouté dans pour pouvoir exprimer les relation entre
les modules.

\section{Formes spéciale}

Les espaces de nom sont gérer avec une forme spécial propre à Gambit. Cet forme
se nomme \lstcode{##namespace}.  La seul restriction sur les espaces de nom
dans Gambit est qu'ils doit se terminer par le caractère \lstcode{#}.
La forme \lstcode{##namespace} permet de préfixer et renommé des identifiants
dans un contexte.

Cela permet de créer un espace distinct pour chaque module. Cela permet
d'éviter les conflits de nom entre les modules tant que les espaces de nom sont
différents. Chaque module commence par déclarer son espace de nom suivit des
définitions des procédures du module. Une déclaration d'espace de nom à la forme
suivante \lstcode{(##namespace ("<name>#"))}. La partie \lstcode{<name>} de la
forme \lstcode{##namespace} doit être un chaine d'au moins un caractères.

\begin{center}
  \begin{tabular}{|l|}
\hline
\begin{mplisting}{0.6}
;; example.scm
(##namespace ("example#" hello))
(define (hello)
  (display "Hello, world!\n"))
(hello)
\end{mplisting}\\\hline
  \end{tabular}
\end{center}
L'exemple ci-dessus est un exemple d'utilisation de la forme \lstcode{##namespace}











Gambit offre un mécanisme pour aider a minimiser les conflits de nom. Ce
mécanisme permet d'associer un identifiant à un autre avec la forme spécial
\lstcode{##namespace}.  L'appel à \lstcode{(##namespace ("foo#" A B))} indique
qu'une référence futur à \lstcode{A} devient une référence à
\lstcode{foo#A} et l'un à \lstcode{B} devient une à \lstcode{foo#B}. Le espace
de nom dans lequel \lstcode{A} et \lstcode{B} est \lstcode{foo#}.

\begin{center}
  \begin{figure}[h]
  \begin{tabular}{|l|}
\hline
\begin{mplisting}{0.5}
;; math#.scm
(##namespace ("math#" fact fib))
\end{mplisting} \\\hline
\begin{mplisting}{0.6}
;; math.scm
(##namespace ("math#" fact fib))
(define (fib n)
  (if (< n 2)
    n
    (+ (fib (- n 1)) (fib (- n 2)))))
(define (fact n)
  (if (< n 2)
    1
    (* n (fact (- n 1)))))
\end{mplisting}\\\hline
  \end{tabular}
  \caption{Écriture d'un petit module mathématique qui implémente les fonctions \lstcode{fact}
    et \lstcode{fib}. Ce module est séparé en 2 fichiers, \texttt{math\#.scm} est un fichier
    contenant les déclarations de l'espace de noms et des définitions de macros que le module
    exporte.}
  \label{fig:math_module1}
\end{figure}
\end{center}
% XXX: END implementation

%\begin{center}
%  \begin{figure}[h]
%  \begin{tabular}{|l|l|}
%\hline
%\begin{mplisting}{0.5}
%;; Library
%(library (math)
%  (export fact)
%  (import (rnrs base))
%  (define (fact n)
%    (if (< n 2)
%      1
%      (* n (fact (- n 1))))))
%\end{mplisting} &
%\begin{mplisting}{0.5}
%;; Main program
%(import
%  (rnrs base)
%  (rnrs io simple)
%  (math))
%
%(display (fact 5))
%(newline)
%\end{mplisting}\\\hline
%  \end{tabular}
%\caption{À gauche, il y a un exemple d'une bibliothèque mathématique dans le format R6RS qui implémente
%la fonction factoriel. À droite, un exemple d'importation de la bibliothèques qui utilise la forme
%spéciale \texttt{import}.}
%\end{figure}
%\end{center}

\section{Implémentation des modules systèmes}

\section{Implémentation des modules R7RS}
% XXX: Implémentation
