
\chapter{Implémentation des modules}

Les format des modules R7RS construit par \lstcode{define-library} contient
plusieurs composantes. L'espace de nom du module qui regroupe toutes les
fonctionnalité et permet de les distinguer des autre modules.  La liste des
importations qui permet d'exprimer les dépendances entre les modules. La liste
des symbole exporté par le module. Pour exprimé les relations entre les modules
certaines forme spéciale plus simple ont été ajouté dans Gambit.  Les modules
R7RS construit par \lstcode{define-library} permettent de spécifier les
identifiants d'un module durant l'importation et aussi durant l'exportation.


% Le format des modules nécessite plusieurs composantes.
% Chaque module a son propre espace de nom
% disjoint des autres modules. Un liste d'importation
% qui indique les dépendances entre les modules et permet
% l'interaction entre les modules. Certaines formes spéciales
% ont été ajouté dans pour pouvoir exprimer les relation entre
% les modules. Les modules R7RS permette de spécifier les identifiant
% d'un module et de les renommer. Les relations entre les modules doivent
% pouvoir être exprimées.  La liste des modules requis par le module
% courant et les modules publiés par le module courant.

% - Créer un espace pour les modules.
% - Spécifier les dépendence. (2)
% - Spécifier les paramètre de compilation.

\section{Formes spéciales}

Les espaces de nom sont gérer avec une forme spécial propre à Gambit. Cet forme
se nomme \lstcode{##namespace} et permet associer des identifiant à un espace
de nom.  Un espace de nom est composé de n'importe quel caractère, il doit
seulement être terminé par le caractère \lstcode{#}.  La forme
\lstcode{##namespace} permet aussi d'associer un identifiant à un autre
identifiant dans un espace de nom donné.


La liste des modules des requis par un module est donnée par plusieurs
invocations de la forme spéciale \lstcode{##demand-module}. Cette forme
accepte en argument le nom de la dépendance. Le fonctionnement de cette
forme est similaire à la procédure \lstcode{load} sauf qu'elle gère le
chargement multiple et est expression \textit{toplevel}.
\begin{center}
  \begin{mplisting}{0.9}
(##demand-module A)
(##demand-module A)
(##demand-module B)
\end{mplisting}
\end{center}

Les noms des modules exporté par une module est spécifier par la forme
spéciale \lstcode{##supply-module}. Elle accepte comme paramètre le nom
du module exporté.


Gambit offre un mécanisme pour aider a minimiser les conflits de nom. Ce
mécanisme permet d'associer un identifiant à un autre avec la forme spécial
\lstcode{##namespace}.  L'appel à \lstcode{(##namespace ("foo#" A B))} indique
qu'une référence futur à \lstcode{A} devient une référence à
\lstcode{foo#A} et l'un à \lstcode{B} devient une à \lstcode{foo#B}. Le espace
de nom dans lequel \lstcode{A} et \lstcode{B} est \lstcode{foo#}.

\begin{center}
  \begin{figure}[h]
  \begin{tabular}{|l|}
\hline
\begin{mplisting}{0.5}
;; math#.scm
(##namespace ("math#" fact fib))
\end{mplisting} \\\hline
\begin{mplisting}{0.6}
;; math.scm
(##namespace ("math#" fact fib))
(define (fib n)
  (if (< n 2)
    n
    (+ (fib (- n 1)) (fib (- n 2)))))
(define (fact n)
  (if (< n 2)
    1
    (* n (fact (- n 1)))))
\end{mplisting}\\\hline
  \end{tabular}
  \caption{Écriture d'un petit module mathématique qui implémente les fonctions \lstcode{fact}
    et \lstcode{fib}. Ce module est séparé en 2 fichiers, \texttt{math\#.scm} est un fichier
    contenant les déclarations de l'espace de noms et des définitions de macros que le module
    exporte.}
  \label{fig:math_module1}
\end{figure}
\end{center}
% XXX: END implementation

%\begin{center}
%  \begin{figure}[h]
%  \begin{tabular}{|l|l|}
%\hline
%\begin{mplisting}{0.5}
%;; Library
%(library (math)
%  (export fact)
%  (import (rnrs base))
%  (define (fact n)
%    (if (< n 2)
%      1
%      (* n (fact (- n 1))))))
%\end{mplisting} &
%\begin{mplisting}{0.5}
%;; Main program
%(import
%  (rnrs base)
%  (rnrs io simple)
%  (math))
%
%(display (fact 5))
%(newline)
%\end{mplisting}\\\hline
%  \end{tabular}
%\caption{À gauche, il y a un exemple d'une bibliothèque mathématique dans le format R6RS qui implémente
%la fonction factoriel. À droite, un exemple d'importation de la bibliothèques qui utilise la forme
%spéciale \texttt{import}.}
%\end{figure}
%\end{center}

\section{Implémentation des modules systèmes}

Cette forme est utilisé affin de créer un espace distinct pour chaque module.
Cela permet d'éviter les conflits de nom entre les identifiants utilisé au sein
des modules.  Chaque module commence par déclarer son espace de nom suivit des
définitions des procédures du module.  Une déclaration d'espace de nom à la
forme suivante \lstcode{(##namespace ("<name>#"))}. La partie \lstcode{<name>}
de la forme \lstcode{##namespace} doit être un chaine d'au moins un caractères.
Toutes les caractères sont permis.

\begin{center}
  \begin{tabular}{|l|}
\hline
\begin{mplisting}{0.6}
;; example.scm
(##namespace ("example#" hello))
(define (hello)
  (display "Hello, world!\n"))
(hello)
\end{mplisting}\\\hline
  \end{tabular}
\end{center}
L'exemple ci-dessus est un exemple d'utilisation de la forme \lstcode{##namespace}
dans pour créer un espace pour le module \lstcode{example}. La procédure \lstcode{hello}
est définit dans l'espace de nom \lstcode{example#}.

\section{Implémentation des modules R7RS}
% XXX: Implémentation
