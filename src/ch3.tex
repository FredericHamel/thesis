
\chapter{Implémentation des modules}
\label{ch:modules_implementation}

Ce chapitre traite de notre implémentation des modules qui se veut portable
entre les différentes implémentations de Scheme.  Notre approche des modules se
base sur la syntaxe de R7RS.  Notre implémentation des modules utilise des
formes déjà existantes et aussi de nouvelles formes qui ont été ajoutées au
besoin.  Ces nouvelles formes nous ont permis d'intégrer la syntaxe des modules
R7RS avec les modules primitifs de Gambit.

Les formats des modules R7RS, construits avec \lstcode{define-library}, contiennent
plusieurs composantes. L'espace de nom du module qui regroupe toutes les
fonctionnalités. Une liste des modules qui sont utilisés par le module courant.
Une liste des symboles exportés par le module. Il est possible de modifier les
identifiants d'un module lors de l'importation et de l'exportation.
L'importation multiple d'un module doit correspondre à un seul chargement.
Pour exprimer les relations entre les modules certaines formes spéciales
ont été ajoutées dans Gambit.  Si les concepteurs de bibliothèques respectent
la syntaxe R7RS, alors il est possible de l'importer dans Gambit. C'est indépendant
du système dans lequel les bibliothèques ont été écrites.



% Le format des modules nécessite plusieurs composantes.
% Chaque module a son propre espace de nom
% disjoint des autres modules. Un liste d'importation
% qui indique les dépendances entre les modules et permet
% l'interaction entre les modules. Certaines formes spéciales
% ont été ajouté dans pour pouvoir exprimer les relation entre
% les modules. Les modules R7RS permette de spécifier les identifiant
% d'un module et de les renommer. Les relations entre les modules doivent
% pouvoir être exprimées.  La liste des modules requis par le module
% courant et les modules publiés par le module courant.

% - Créer un espace pour les modules.
% - Spécifier les dépendence. (2)
% - Spécifier les paramètre de compilation.

\section{La forme \lstcode{\#\#namespace}}

Les espaces de nom sont gérés avec une forme spéciale propre à Gambit. Cette
forme se nomme \lstcode{##namespace} et permet d'associer des identifiants à
d'autres identifiants.  Cette forme primitive est présente dans Gambit depuis longtemps.  Un
espace de nom se compose de n'importe quelle séquence de caractère terminé par
un \lstcode{#}. Il y a seulement l'espace de nom vide qui ne respecte pas cette
règle, c'est l'espace de nom par défaut.  Les associations de symbole données
par la forme \lstcode{##namespace} respectent la portée lexicale. Il y a trois
types d'opérations avec les espaces de nom.

Il y a les espaces de nom global qui s'applique à tous les symboles qui ne
contiennent pas de \lstcode{#}. \\
\begin{figure}[ht]
  \centering
  \lstset{language={scheme},frame=single}
  \begin{mplisting}{0.6}
(##namespace ("<ns>"))
;; <symbol-name> => <ns><symbol-name>
\end{mplisting}
  \caption{Namespace Global}
  \label{fig:forms->namespace-global}
\end{figure}

Il est possible de spécifier la liste des symboles qui sont affectés par la
déclaration d'espace de nom. À partir de la syntaxe du
\ref{fig:forms->namespace-global} il suffit d'ajouter les symboles après le nom
de l'espace de nom.\\
\begin{figure}[ht]
  \centering
  \lstset{language={scheme},
          frame=single}
  \begin{mplisting}{0.6}
(##namespace ("<ns>" A B ...))
;; A => <ns>A
;; B => <ns>B
;; ...
\end{mplisting}
  \caption{Namespace Set}
  \label{fig:forms->namespace-set}
\end{figure}

%Chaque symbole dans l'espace de nom peut être renommé à un nouveau symbole en
%spécifier une pair contenant l'association. La forme ressemble à l'exemple
La forme \lstcode{##namespace} permet aussi d'associer un identifiant à un
autre identifiant dans un espace de nom donné. Chaque association est marquée
par une paire qui alias le premier élément par le second. Par exemple, la paire
\lstcode{(<old> <new>)} remplace \lstcode{<old>} par \lstcode{<new>}.\\
\begin{figure}[ht]
  \centering
  \lstset{language={scheme},
          frame=single}
  \begin{mplisting}{0.6}
(##namespace ("<ns>" (<old> <new>) ...))
;; <old> => <ns><new>
;; ...
\end{mplisting}
  \caption{Namespace Rename}
  \label{fig:forms->namespace-rename}
\end{figure}

% La syntaxe de la
% forme \lstcode{##namespace} qui inclus toutes les symboles dans l'espace de nom
% marqué par \lstcode{<ns>} est:

% \begin{center}
%   \begin{mplisting}{0.8}
% (##namespace ("<ns>"))
% \end{mplisting}
% \end{center}

% Il est possible de spécifier une liste des symboles qui appartiennent à
% l'espace de nom. Notez que seulement le espace de nom est entre guillemets.

% \begin{center}
%   \begin{mplisting}{0.8}
% (##namespace ("<ns>" sym1 ...))
% \end{mplisting}
% \end{center}


% L'utilisation de la forme \lstcode{##namespace} permet de associer plusieurs
% identifiant dans un espace de nom spécifique.  La procédure \lstcode{hello} est
% défini dans l'espace de nom \lstcode{hello#}.

Cette forme est utilisée pour créer un espace distinct pour chaque module.
Cela permet d'éviter les conflits de nom entre les identifiants des modules.
Chaque module commence par déclarer son espace de nom suivi des définitions
des procédures du module. Les différentes formes d'espace de noms sont données
par les figures \ref{fig:forms->namespace-global}, \ref{fig:forms->namespace-set}
et \ref{fig:forms->namespace-rename}.
\\
\begin{figure}[ht]
  \centering
  \lstset{language={scheme},
          frame=single}
\begin{mplisting}{0.6}
;; hello.scm
(##namespace ("hello#" hello))
(define (hello)
  (display "Hello, world!\n"))
(hello)
\end{mplisting}
  \caption{Module Hello}
  \label{fig:namespace->hello}
\end{figure}

L'exemple \ref{fig:namespace->hello} est un exemple d'utilisation de la forme
\lstcode{##namespace} pour créer un espace pour le module \lstcode{hello}.
La procédure \lstcode{hello} est dans l'espace de nom \lstcode{hello#}.

\section{La forme \lstcode{\#\#demand-module} et \lstcode{\#\#supply-module}}

Le mécanisme de chargement des modules est géré par la forme spéciale
\lstcode{##demand-module}. Cette forme indique au système de charger un module
s'il n'est pas déjà chargé. Cette forme gère le chargement multiple d'un
module. Elle est utilisée pour importer la liste des modules requis par le
module courant.  Le fonctionnement de cette forme est similaire à la procédure
\lstcode{load} avec quelques différences. La forme \lstcode{##demand-module} est
une macro qui génère une expression vide. L'effet de cette forme agit après la
phase d'expansion des macros. Le paramètre passé à \lstcode{##demand-module}
doit être un symbole qui correspond au nom du module. La procédure \lstcode{load}
requiert le chemin complet vers le fichier à charger.

Il est à noter que l'ordre des \lstcode{##demand-module} correspond à l'ordre
dont les modules sont visités. Cette forme agit, peu importe son emplacement
où l'expansion de macros est appliquée.

%% Review.

% \begin{center}
%   \begin{mplisting}{0.9}
% (##demand-module A)
% (##demand-module A)
% (##demand-module B)
% \end{mplisting}
% \end{center}

Une forme spéciale conjointe au \lstcode{##demand-module}, qui indique le nom
symbolique des modules importés des modules exportés. Cette forme spéciale est
\lstcode{##supply-module}, elle accepte comme paramètre le nom du module exporté
par l'entité courant.  La syntaxe de ces deux formes dans la figure
\ref{fig:syntax->demand/supply-module}.\\
\begin{figure}[ht]
  \centering
  \lstset{language={scheme},
          frame=single}
  \begin{mplisting}{0.8}
(##demand-module %*\textit{<module-ref>}*)
(##supply-module %*\textit{<module-ref>}*)
\end{mplisting}
  \caption{Syntaxe \lstcode{demand-module} et \lstcode{supply-module}}
  \label{fig:syntax->demand/supply-module}
\end{figure}

\subsection{Les méta informations}
Un module a des informations qui sont utilisées lors de l'expansion
et même la compilation. Ces informations sont spécifiées par la forme
\lstcode{##meta-info}. Cette forme accepte au moins un paramètre qui
correspond au nom de la méta information, le reste des paramètres est la valeur
associer à la méta donnée.

\begin{center}
  \begin{mplisting}{0.5}
(##meta-info %*\textit{<name>}* %*\textit{<value>}*)
(##meta-info %*\textit{<name>}* %*\textit{<value>}* ...)
\end{mplisting}
\end{center}

Les méta informations sont utilisées pour donner des paramètres de compilation
du module. Les différentes méta informations sont \lstcode{cc-options},
\lstcode{ld-options}, \lstcode{ld-options-prelude}, \lstcode{pkg-config}
et \lstcode{pkg-config-path}. Ces méta informations ne sont utiles que pour les modules
compilés.

\begin{itemize}
  \item Les \lstcode{cc-options} sont ajoutés aux options de la commande qui
    invoque le compilateur C.

  \item Les méta informations \lstcode{ld-options} et \lstcode{ld-options-prelude}
    composent les paramètres de la commande qui invoque l'éditeur de lien.
    Les paramètres dans \lstcode{ld-options-prelude} précèdent ceux
    qui sont dans \lstcode{ld-options}.

  \item \lstcode{pkg-config} contient le nom des bibliothèques C à être
    lié au module Scheme. Les options nécessaires pour au compilateur
    C sont déterminés automatiquement par l'utilitaire \lstcode{pkg-config}.

  \item \lstcode{##pkg-config-path} ajoute des répertoires à la variable d'environnement
    \lstcode{PKG_CONFIG_PATH} qui est utilisée par l'utilitaire \lstcode{pkg-config}.

\end{itemize}

%% XXX: antidote here
\section{Implémentation des modules primitifs}

Un module primitif est généralement constitué d'un fichier d'entête avec la
déclaration de l'espace de nom et les définitions de macros et un fichier
contenant les procédures. Dans Gambit les fichiers d'entête sont marqués par un
\lstcode{#} juste avant l'extension.

\begin{itemize}
  \item \texttt{\textit{<name>}\#.scm} est la structure du nom fichier d'entête.
    Ce fichier contient des déclarations d'espace de nom et des
    définitions de macros.

  \item \texttt{\textit{<name>}.scm} est la structure du nom du fichier qui contient
    les procédures du module.

\end{itemize}

Le nom des fichiers doit correspondre à la dernière partie du nom de module.  Par
exemple, le module primitif \lstcode{angle2} doit inclure les fichiers
\lstcode{angle2/angle2.scm} et généralement \lstcode{angle2/angle2#.scm}.\\
\begin{figure}[ht]
  \lstset{language={scheme},frame=single}
\begin{mplisting}{0.9}
;; angle2/angle2#.scm
(##namespace ("angle2#" deg->rad rad->deg))
\end{mplisting}\\[4ex]
\begin{mplisting}{0.9}
;; angle2/angle2.scm
(include "angle2#.scm")
(##namespace ("angle2#" factor))
(##supply-module angle2)
(define factor (/ (atan 1) 45))
(define (deg->rad x) (* x factor))
(define (rad->deg x) (/ x factor))
\end{mplisting}
  \caption{Écriture d'un module qui implémente une pile. Ce module est
           séparé en 2 fichiers. Le fichier \texttt{stk\#.scm} qui contient les
           exportations et \texttt{stk.scm} qui contient les implémentations des
           fonctions.}
  \label{fig:module->angle}
\end{figure}

% XXX: END implementation
%\todo{Continuer ici}

%\begin{center}
%  \begin{figure}[h]
%  \begin{tabular}{|l|l|}
%\hline
%\begin{mplisting}{0.5}
%;; Library
%(library (math)
%  (export fact)
%  (import (rnrs base))
%  (define (fact n)
%    (if (< n 2)
%      1
%      (* n (fact (- n 1))))))
%\end{mplisting} &
%\begin{mplisting}{0.5}
%;; Main program
%(import
%  (rnrs base)
%  (rnrs io simple)
%  (math))
%
%(display (fact 5))
%(newline)
%\end{mplisting}\\\hline
%  \end{tabular}
%\caption{À gauche, il y a un exemple d'une bibliothèque mathématique dans le format R6RS qui implémente
%la fonction factoriel. À droite, un exemple d'importation de la bibliothèques qui utilise la forme
%spéciale \texttt{import}.}
%\end{figure}
%\end{center}

Dans l'exemple \ref{fig:module->angle}, il y a dans \lstcode{angle2/angle2.scm}
l'inclusion de du fichier d'entête \lstcode{angle2/angle2#.scm} qui ajoute une
déclaration redondante de l'espace de nom dans ce cas.  La déclaration
\lstcode{(##namespace ("angle2#"))} implique l'espace de nom ajouté par
l'inclusion du fichier d'entête. Il est possible que l'espace de nom déclaré
dans \lstcode{angle2/angle2#.scm} ne corresponde pas à celui utilisé dans
\lstcode{angle2/angle2.scm}.

La forme \lstcode{##namespace} dans l'exemple \ref{fig:module->angle}
s'applique aux identifiants suivants:
\begin{center}
  \lstset{language={scheme},keepspaces=true}
  \begin{mplisting}{0.3}
factor    --> stk#factor
deg->rad  --> stk#deg->rad
rad->deg  --> stk#rad->deg
\end{mplisting}
\end{center}

\subsection{La forme \lstcode{\#\#import}}
%
L'importation des modules est effectuée par la forme \lstcode{##import} qui
effectue deux actions, l'inclusion du fichier \lstcode{<name>#.scm} et un
chargement des définitions.  La forme \lstcode{##import}, comme
\lstcode{##demand-module} s'occupe de trouver l'emplacement du fichier d'entête
à partir du nom du module. Elle génère le \lstcode{##include} du fichier
d'entête s'il existe et un \lstcode{##demand-module} du module.  L'importation
\lstcode{(##import angle2)} est équivalente à:\\
\begin{figure}[ht]
  \centering
  \lstset{language={Scheme},frame=single}
  \begin{mplisting}{0.9}
(##include "/un/chemin/angle2/angle2#.scm")
(##demand-module stk)
\end{mplisting}
  \caption{Expansion de \lstcode{(\#\#import angle2)}}
  \label{fig:prim-import->stk}
\end{figure}


\section{Implémentation des modules R7RS}
% XXX: Implémentation
Pour que le système de module soit compatible avec d'autres implémentations
Scheme,  les modules haut niveau sont définis dans le standard R7RS
Small~\cite{Scheme:R7RS}. Les modules sont définis par la forme
\lstcode{define-library} la syntaxe est donnée par la
figure~\ref{fig:syntax->define-library}. Les formes \lstcode{define-library} et
\lstcode{import} utilise les formes spéciales utilisées par les modules
primitifs. L'élément qui distingue un module primitif et un module R7RS
est l'utilisation de la forme \lstcode{define-library}.

\subsection{Expansion du \lstcode{import}}
\label{sec:import-expand}

La façon que la forme \lstcode{import} est expansée dépend du type de module qui
est importé. L'importation d'un module primitif est différente de l'importation
d'un module R7RS. Gambit permet l'importation d'un module primitif en utilisant
la même forme que pour les modules R7RS. Les capacités du \lstcode{import}
dépendent s'il est d'un \lstcode{define-library} d'un programme principal. Dans
le cas d'un \lstcode{define-library} le \lstcode{import} supporte l'importation
relative, qui est une extension de Gambit.


\subsubsection{Importation d'un module primitif}
L'importation d'un module primitif limite la syntaxe du \lstcode{import}.
Il n'est pas possible d'utiliser les extensions \lstcode{only}, \lstcode{except}
et \lstcode{rename} sur un module primitif présentement. Le \lstcode{import} R7RS se rabat
sur le \lstcode{##import} des modules primitifs qui ne supporte pas les extensions R7RS.
Cela permet d'utiliser des modules primitifs dans un contexte R7RS.\\
\begin{figure}[ht]
  \lstset{language={scheme},frame=single}
\begin{mplisting}{0.8}
;; expansion of (import (termite))
(##import termite)
\end{mplisting}
    \caption{Expansion du \texttt{import} d'un module primitif}
    \label{fig:import->expand-primitive}
\end{figure}

\subsubsection{Importation d'un module R7RS}

L'importation d'un module R7RS est expansée en au plus trois parties.  Un
\lstcode{##demand-module} qui s'occupe de charger l'implémentation des
procédures du module. Une déclaration d'espace de nom qui donne accès aux
identifiants que le module exporte. L'implémentation des macros qui sont
exportées du module.

L'instruction de chargement du module est générée dans tous les cas qu'un
module définit des procédures. Un module qui ne définit que des macros ne
nécessite pas d'être chargé durant l'exécution seulement dans le contexte
d'expansion des macros. L'importation d'un module R7RS qui ne contient qu'une
déclaration \lstcode{export} ne nécessite pas d'être chargé durant l'exécution.
Ce type de module est utilisé pour exporter les fonctionnalités déjà
implémentées dans Gambit dans un contexte R7RS.

La forme utilisée pour rendre disponible l'ensemble des identifiants importés
est \lstcode{##namespace}. L'ensemble des identifiants importés dépend de la
forme du \lstcode{import}. Par défaut, tous les identifiants exportés par le
module sont importés.  Les opérateurs \lstcode{only} et \lstcode{except}
affectent le nombre d'identifiants importés. Les opérateurs \lstcode{prefix} et
\lstcode{rename} affecte le nom des identifiants.  Dans l'exemple
\ref{fig:import->expand-r7rs}, l'importation inclut l'ensemble des identifiants
exportés par le module. L'ensemble des formes \lstcode{##namespace} générées par
un \lstcode{import} est donné par la
figure~\ref{fig:import->expand-r7rs-namespace}.
\\

\begin{figure}[ht]
  \centering
  \lstset{language={scheme},frame=single}
  \begin{mplisting}{1}
;; expansion of (import (github.com/gambit/hello))
(##demand-module github.com/gambit/hello)
(##namespace ("github.com/gambit/hello#" hi salut))
;; macros
\end{mplisting}
  \caption{L'exemple de l'expansion du \texttt{import} du module R7RS
    \lstcode{github.com/gambit/hello} qui exporte les procédures
    \lstcode{hello} et \lstcode{hi}}
  \label{fig:import->expand-r7rs}
\end{figure}

% TODO: Différent namespaces généré par import.

\begin{figure}[ht]
  \centering
  \lstset{language={scheme},frame=single}
  \begin{mplisting}{1}
;; (import (only (github.com/gambit/hello) hi))
(##namespace ("github.com/gambit/hello#" hi))

;; (import (except (github.com/gambit/hello) hi))
(##namespace ("github.com/gambit/hello#" salut))

;; expansion of (import (prefix (github.com/gambit/hello) m-))
(##demand-module github.com/gambit/hello)
(##namespace ("github.com/gambit/hello#" (m-hi hi) (m-salut salut)))

;; (import (rename (github.com/gambit/hello) (hi bonjour)))
(##namespace ("github.com/gambit/hello#" (howdy hi) salut))
\end{mplisting}
  \caption{Différent \lstcode{\#\#namespace} généré par
    l'expansion du \texttt{import} d'un module R7RS.}
  \label{fig:import->expand-r7rs-namespace},
\end{figure}


\subsection{Expansion du \lstcode{define-library}}

La forme \lstcode{define-library} est expansé dans les formes qui composent un
module primitif. Chacune des déclarations de la bibliothèque est utilisée dans
l'expansion du \lstcode{define-library}. La déclaration d'exportation est
valide si tous les identifiants exportés sont distincts. Une déclaration
\lstcode{export} qui exporte un identifiant plusieurs fois cause une erreur de
syntaxe. Les informations sur les identifiants exportés ne sont pas utilisé lors
l'expansion du \lstcode{define-library}, mais lors de l'importation de cette
bibliothèque.  Les déclarations \lstcode{import} sont expansées de la même
façon que dans contexte des programmes principaux.\\

\begin{figure}[ht]
  \centering
  \lstset{language={scheme},frame=single}
\begin{mplisting}{0.9}
(define-library (hello)
  (import (scheme base) (scheme write))
  (export hi salut)
  (begin
    (define (exclaim msg1 msg2)
      (display msg1)
      (display msg2)
      (display "!\n"))
    (define (hi name) (exclaim "hello " name))
    (define (salut name) (exclaim "bonjour " name))
    ;; it is best for a library to not have side-effects...
    #;(salut "le monde")))
\end{mplisting}
  \caption{C'est le code source du module
    \lstcode{github.com/gambit/hello} avant l'expansion.}
  \label{fig:define-library->expand}
\end{figure}
  \vspace*{4ex}
\begin{figure}[ht]
  \centering
  \lstset{language={scheme},frame=single}
  \begin{mplisting}{0.9}
;; expansion of (define-library (hello) ...)
(##declare (block))
(##supply-module github.com/gambit/hello)
(##namespace ("github.com/gambit/hello#"))
(##namespace ("" define ...))
(##namespace ("" write-shared write display write-simple))
(define (exclaim msg1 msg2)
    (display msg1) (display msg2) (display "!\n"))
(define (hi name) (exclaim "hello " name))
(define (salut name) (exclaim "bonjour " name))
(##namespace (""))
\end{mplisting}
  \caption{Expansion de la forme \lstcode{define-library}
    du module \lstcode{github.com/gambit/hello}.}
  \label{fig:define-library->expand-after}
\end{figure}

\subsubsection{Extensions de Gambit}

Gambit offre des extensions au \lstcode{define-library} et au \lstcode{import}.
L'importation dans le contexte d'une bibliothèque peut être relative au module
courant. Plusieurs déclarations supplémentaires ont été ajoutées dans la forme
\lstcode{define-library}.

\begin{itemize}
  \item \lstcode{namepase}
  \item \lstcode{cc-options}
  \item \lstcode{ld-options} et \lstcode{ld-options-prelude}
  \item \lstcode{pkg-config} et \lstcode{pkg-config-path}
\end{itemize}


La figure~\ref{fig:relative-import} est un exemple d'importation
relatif.  L'importation relative part:w
du \lstcode{module-ref} du module courant.
Un \lstcode{import} de \lstcode{(.. C)} à partir du module \lstcode{(A B)}
correspond à l'importation de \lstcode{(A C)}. C'est pour permettre au sous
module de tests unitaires de référer au module principal en préservant le
\lstcode{<module-ref>} avec la version. \\

\begin{figure}[ht]
  \centering
  \lstset{language={scheme},frame=single}
  \begin{mplisting}{0.8}
(define-library (A B)
  (import (.. C))  ;; => (import (A C))
  (import (..))) ;; => (import (A))
\end{mplisting}
  \caption{Importation relatif du module \lstcode{(A C)}}
  \label{fig:relative-import}
\end{figure}

La déclaration \lstcode{namespace} permet de forcer l'espace de nom d'un module.
L'utilisation primaire de cette déclaration est l'implémentation de modules qui
exporte les fonctionnalités déjà implémentées dans Gambit. \\

\begin{figure}
  \lstset{language={scheme},
          frame=single}
  \begin{mplisting}{0.8}
(define-library (scheme case-lambda)
  (namespace "")
  (export
case-lambda
))
\end{mplisting}
  \caption{Implémentation de la bibliothèque système \lstcode{(scheme case-lambda)}.}
  \label{fig:module->scheme/case-lambda}
\end{figure}

Les déclarations \lstcode{cc-options}, \lstcode{ld-options}, \lstcode{ld-options-prelude},
\lstcode{pkg-config} et \lstcode{pkg-config-path} permet d'ajouter des éléments dans les
méta informations respectifs.

\section{Conclusion}

TODO!!!
