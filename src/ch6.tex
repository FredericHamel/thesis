\chapter{Migration de code}
\label{ch:task_migration}

Les systèmes distribués sont constitué d'un ensemble de nœuds interconnecter
de calculs. Les nœuds interagissent par l'envoie et la réception de message
au sein d'un réseau de communication. Chaque nœud a but spécifique. Le
\textit{World Wide Web} est un exemple notable qui permet d'avoir un aperçu.
Ils est composé de clients et de serveurs qui roulent des application
clients et serveurs différent.

L'implémentation d'un système distribué inclut le développement des applications
installé sur les nœuds et la logique d'interaction entre les nœuds. Il est
possible de voir l'ensemble des programme sur les nœuds comme un programme
global. Les problèmes discutés dans ce chapitre sont les suivant:

\begin{itemize}

  \item {\bf RPC:} Comment l'appelle distant à une procédure (RPC)
    implémenté quand l'envoyeur et le receveur ne sont pas conçu
    ensemble?

  \item {\bf Mise à jour de code:} Comment la mise à jour du programme d'un
    nœud est effectué lors d'un \textit{bugfix} ou une nouvelle
    version est disponible?

  \item {\bf Migration de tâche:} Comment déplacer un service sur un nouveau
    nœud quand le système sous-jacent est sur un système d'exploitation
    différent, a un architecture différente, \dots?

  \item {\bf Opération continue:} Comment éviter les interruptions dans les
    situations précédentes?

\end{itemize}

Le langage Termite Scheme\cite{DBLP:conf/erlang/Germain06} a été conçu
pour simplifier l'implémentation de systèmes distribués et fournit
certaine solution au problèmes de ces système. Le langage Termite Scheme est
fortement inspiré des concepts du langage de programmation d'Erlang avec la
syntaxe et la sémantique de Scheme. Une fonctionnalité intéressante qui est
absent en Erlang est la capacité d'envoyer une continuation en message.
Termite Scheme est implémenté sur le système Gambit Scheme qui offre la
une façon de sérialiser la plupart des objets Scheme incluant les
procédures et les continuations.

La sérialisation de procédures est un outil utile dans pour implémenter
un protocole RPC. En plus des procédures, il est possible de sérialiser
des continuations. Une continuation est une structure de donnée qui capture
l'état d'un processus. Donc il est possible de transmettre l'état d'un
processus sur un autre nœud.

L'implémentation original de Termite avait certaines limitations
lors de la sérialisation des procédures et des continuations. Dans
le cas interprété les procédures et les continuations sont transmis
sans problème entre les nœuds. Dans le cas compilé, il faut que
chaque nœud possède la définition des procédures qui sont transmis.


\section{Le langage Termite}

Ce langage conçu par Guillaume Germain en 2006 est l'implémentation
du style de programmation par message d'Erlang dans Gambit Scheme.
Les processus sont représentés par les threads de Scheme. La communication
entre ces processus est effectué par un système de boîte de message
présent dans Gambit. Chaque thread possède une queue de message entrant.

Termite expose plusieurs procédures pour gérer les processus et la transmission
de message entre chacun des nœuds.

\begin{itemize}
  \item La procédure \lstcode{(spawn thunk #!key name)} permet de
    créer des processus Termite.

  \item La procédure \lstcode{(! node msg)} envoie le message
    \lstcode{msg} au nœud \lstcode{node}.

  \item La procédure \lstcode{(?)} permet de recevoir un message
    envoyé par un autre processus au nœud courant.


  \item La procédure \lstcode{(!? node msg)} envoie un message au
    nœud \lstcode{node} et attend la réponse.
\end{itemize}



