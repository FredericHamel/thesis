\chapter{Gestion des modules}
\label{ch:module_management}

Gambit utilise deux répertoires principaux pour les modules.  Ils sont associés
aux alliasses de répertoire \lstcode{\~\~userlib} et \lstcode{\~\~lib} qui
correspond aux emplacements respectifs des modules utilisateurs et des modules
systèmes. Les différents opérations sont l'installation de modules, la mise à
jour des modules, la désinstallation de modules, l'exécution des tests
unitaires et la compilation. La gestion des modules est faite par le module
\lstcode{gambit/pkg}. Il est possible d'invoquer ces opérations par des
arguments de ligne de commande passé à Gambit.

% \begin{itemize}
%   \item \lstcode{install} effectue l'installation de modules.
%   \item \lstcode{update} effectue la mise à jour de la cache des modules demandés. Cela
%     permet d'actualiser les nouvelles versions disponible d'un module.
%   \item \lstcode{uninstall} désinstalle les modules spécifié.
%   \item \lstcode{test} exécute les tests unitaires des modules spécifiés.
% \end{itemize}

\section{Installation}

Les modules sont hébergés sur des serveur de version tel que github, gitlab,
bitbucket, etc. L'installation des modules s'effectue par l'intermédiaire de
\lstcode{git}. Un module est installé en plusieurs étapes. Tout d'abord, la
branche \lstcode{master} dépôt du module est cloné. Ensuite un archive de la
version est faite et extraite dans le préfixe des modules. Le préfixe est par
défaut \lstcode{\~\~userlib}.  Il est possible de spécifier un préfixe
d'installation dans lequel installer les modules. Plusieurs versions d'un même
module coexistent dans le même préfixe d'installation.

La branche \lstcode{master} est utilisé comme version de développement et
comme cache pour installer les autres versions. Une version d'un module est soit
un commit une branche ou un étiquette. L'installation d'une version
spécifique utilise la cache pour récupérer l'archive de la version demandé
et l'extraire dans l'espace des module.

La procédure \lstcode{install} de \lstcode{gambit/pkg} accepte deux paramètres:
le nom du module et de façon optionnelle le préfixe d'installation. Elle
retourne la valeur de vérité vrai (\lstcode{#t}) si l'installation réussi,
sinon faux (\lstcode{#f}).
\begin{center}
  \begin{mplisting}{0.4}
(install mod #!optional to)
\end{mplisting}
\end{center}

L'installation peut être effectué en passant l'option \lstcode{-install}
à l'interprète gambit. Cette option requière le nom du module et
de façon optionnelle le préfixe d'installation.
\begin{center}
  \begin{mplisting}{0.8}
> gsi -install [-to <path>] module [...]
\end{mplisting}
\end{center}
Le préfixe \lstcode{<path>} est la racine utilisée pour installer les modules
et est spécifier par l'option \lstcode{-to}.  La racine par défaut est
\lstcode{\~\~userlib}. Voici un exemple d'installation d'une version spécifique du module
\lstcode{semver} qui implémente la logique du \textit{semantic versioning}.

\begin{center}
  \begin{mplisting}{1}
> gsi -install -to /tmp/exemple github.com/frederichamel/semver/tree/1.0.1
\end{mplisting}
\end{center}

\section{Désinstallation}

La désinstallation d'un module consiste à supprimer les fichier
de ce module.

Cette commande efface les fichiers liés au module spécifier.
Par défaut, elle affecte les modules installés dans le répertoire
utilisateur. Pour désinstaller un module installé dans un autre
répertoire, il faut le spécifier par l'option \lstcode{-to}. Cette option
est la même que dans l'installation.

\begin{center}
  \begin{mplisting}{0.8}
> gsi -uninstall [-to <path>] module
\end{mplisting}
\end{center}

\section{Mise à jour}
Cette opération actualise la branche \lstcode{master} du module.
Cela donne accès au nouvelle publication d'un module. Pour installer
une nouvelle version d'un module, il suffit de faire la mise à jour
de la branche master et d'installer la nouvelle version.

\begin{center}
  \begin{mplisting}{0.8}
> gsi -update [-to <path>] module
\end{mplisting}
\end{center}

\section{Tests unitaires}
Les tests unitaires exécutés sont dans dans un fichier conjoint au module.
Gambit offre un module de test unitaire nommé \lstcode{gambit/test}. Il
contient plusieurs procédure pour tester le bon fonctionnement d'un module.
Les tests unitaires pour un module nommé \lstcode{A} est un fichier
\lstcode{A-test.scm} dans le répertoire du module.

\begin{center}
  \begin{mplisting}{0.8}
> gsi -test [-to <path>] module
\end{mplisting}
\end{center}

La commande de test ci-dessus réussi si l'exécution des teste termine sans
lancer d'erreurs.

% Exemple de tests.

\section{Compilation d'un module}
Il n'y a pas d'options spécial pour demander la compilation d'un module.
Il suffit d'invoquer le compilateur de Gambit avec le nom du module
à compiler. Le nom du module est le même que celui utilisé dans la
l'importation.

%% Module avec du C.



