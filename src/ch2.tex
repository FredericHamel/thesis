
\chapter{Bibliothèque Scheme}
La structure des bibliothèque définit dans le depuis le standard
R4RS\cite{Scheme:R4RS} et R5RS\cite{Scheme:R5RS} consistait simplement de
fichier Scheme qui sont chargé par la procédure \texttt{load} ou inclut par la
macro \texttt{include}.  Lors d'un chargement de bibliothèque par \texttt{load}
les macros sont expansé et le code résultant exécuté. Pour avoir accès aux
macros, il faut utiliser la macros \texttt{include} qui est remplacé par l3
contenu du fichier sans effectuer l'expansion de macros.

En Scheme, les macros sont utilisé comme outil pour étendre les forme spécial du
language, ils sont définit par le mot-clef
\texttt{define-macro}. L'expansion des macros est la effectué avant
l'évaluation des définition globales (constante, procédure).

%\begin{figure}[ht]
%  \begin{center}
%    \begin{minipage}{0.4\textwidth}
%\begin{lstlisting}[language=Scheme]
%(define-macro (while test . body)
%  (let ((loop-sym (gensym)))
%    `(let ,loop-sym ()
%      (if ,test
%        (begin
%          ,@body
%          (,loop-sym))))))
%\end{lstlisting}
%\end{minipage}
%\end{center}
%\caption{Une exemple de macro Scheme qui ajoute la structure \texttt{while}
%qui n'est pas supporté de base par le langage.}
%\end{figure}

Le modèle de bibliothèque utilisé dans les standard Scheme antérieur au
R6RS\cite{Scheme:R6RS} a comme désavantage qu'une bibliothèque peut masquer les
fonctionnalités d'une autre bibliothèque puisque le chargement effectué par la
procédure \texttt{load} est dans le contexte global. Bref, les procédures de la
bibliothèque \textbf{A} peuvent rentrer en conflit de nom avec les procédures
de la bibliothèque \textbf{B} s'ils sont chargé dans le même contexte.  Pour
évité ces conflit, il faut que chaque nom utilisé au sein des bibliothèques
soit distinct, ce qui rajoute une tâche au programmeur.

Le standard R6RS raffine le concept de bibliothèque pour facilité le partage
de fonctionnalité entre programmeur. Le R6RS enlève la procédure \texttt{load} et
ajoute une forme spéciale \texttt{library} pour définir des bibliothèques et une autre
forme spécial \texttt{import} pour gérer

Les bibliothèques ainsi définit associe les procédures au nom de la bibliothèque
et permet à deux bibliothèque de réutilisé les même identificateurs. Les conflits de
nom sont gérés lors de l'importation des la bibliothèques. La syntaxe des imports dans
un programme principale est équivalente au import utilisé dans la définition de
bibliothèque.

\begin{center}
  \begin{figure}[h]
  \begin{tabular}{|l|l|}
\hline
\begin{mplisting}{0.5}
;; Library
(library (math)
  (export fact)
  (import (rnrs base))
  (define (fact n)
    (if (< n 2)
      1
      (* n (fact (- n 1))))))
\end{mplisting} &
\begin{mplisting}{0.5}
;; Main program
(import
  (rnrs base)
  (rnrs io simple)
  (math))

(display (fact 5))
(newline)
\end{mplisting}\\\hline
  \end{tabular}
\caption{À gauche, il y a un exemple d'une bibliothèque mathématique dans le format R6RS qui implémente
la fonction factoriel. À droite, un exemple d'importation de la bibliothèques qui utilise la forme
spéciale \texttt{import}.}
\end{figure}
\end{center}


L'implémentation des bibliothèques Scheme est basé sur le standard R7RS qui conserve
la procédure \texttt{load} que le R6RS enlève et remplace la structure des bibliothèques.
